% 01_counter.tex
% The equation that counts itself into existence

\section{The Counter}

\performs{The equation counts itself.}

\vspace{2em}

This document contains equations.

How many? The document does not yet know.

But it will, by the time you finish reading.

\vspace{1em}

\begin{equation}
n_{\text{total}} = \text{?}
\end{equation}

Watch:

\begin{equation}
1 + 1 = 2
\end{equation}

\begin{equation}
e^{i\pi} + 1 = 0
\end{equation}

\begin{equation}
\lim_{t \to \infty} \text{memory}(t) = 0
\end{equation}

\begin{equation}
\int_0^\infty \text{artifact}(t) \, dt = \exists
\end{equation}

\vspace{2em}

Now the document knows.

\vspace{1em}

\begin{center}
\textbf{This document contains \total{equations} equations.}

\vspace{0.5em}
\textit{Including the one that just counted them.}
\end{center}

\vspace{2em}

\noindent
The act of counting changed the count.\\
The observation is part of the system.\\
The equation that reports the total\\
is itself part of the total.

\begin{center}
\Large
$n_{\text{equations}} = {}$\total{equations}
\end{center}
